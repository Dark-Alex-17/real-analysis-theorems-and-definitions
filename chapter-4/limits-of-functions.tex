\section{Limits of Functions}

\begin{definition}
	Let $A \subseteq \R$. A point $c \in \R$ is a \textbf{cluster point} of $A$ if for every $\delta >0$ there exists at least one point $x \in A,\ x \neq c$ such that $|x-c|<\delta$.
	\\\\This definition is rephrased in the language of neighborhoods as follows: A point $c$ is a cluster point of the set $A$ if every $\delta$-neighborhood $V_\delta (c)=(c-\delta, c+\delta)$ of $c$ contains at least one point of $A$ distinct from $c$.
\end{definition}

\begin{theorem}
	A number $c \in \R$ is a cluster point of a subset $A$ of $\R$ if and only if there exists a sequence $(a_n)$ in $A$ such that $\lim (a_n) = c$ and $a_n \neq c$ for all $n \in \N$.
\end{theorem}

\begin{definition}
	Let $A \subseteq \R$, and let $c$ be a cluster point of $A$. For a function $f:A \rightarrow \R$, a real number $L$ is said to be a \textbf{limit of $f$ at $c$} if, given any $\varepsilon>0$, there exists a $\delta>0$ such that if $x \in A$ and $0 < |x-c|<\delta$, then $|f(x)-L|<\varepsilon$.
\end{definition}

\begin{theorem}
	If $f:A \rightarrow \R$ and if $c$ is a cluster point of $A$, then $f$ can have only one limit at $c$.
\end{theorem}

\begin{theorem}
	Let $f:A \rightarrow \R$ and let $c$ be a cluster point of $A$. Then the following statements are equivalent.
	\begin{enumerate}
		\item $\lim\limits_{x\to c}=L$.
		\item Given any $\varepsilon$-neighborhood $V_\varepsilon (L)$ of $L$, there exists a $\delta$-neighborhood $V_\delta (c)$ of $c$ such that if $x \neq c$ is any point in $V_\delta (c) \cap A$, then $f(x)$ belongs to $V_\varepsilon (L)$.
	\end{enumerate}
\end{theorem}

\begin{theorem}[\textbf{Sequential Criterion}]
	Let $f:A \rightarrow \R$ and let $c$ be a cluster point of $A$. Then the following are equivalent.
	\begin{enumerate}
		\item $\lim\limits_{x\to c} f=L$.
		\item For every sequence $(x_n)$ in $A$ that converges to $c$ such that $x_n\neq c$ for all $n \in \N$, the sequence $(f(x_n))$ converges to $L$.
	\end{enumerate}
\end{theorem}

\begin{theorem}[\textbf{Divergence Criteria}]
	Let $A \subseteq \R$, let $f:A \rightarrow \R$ and let $c \in \R$ be a cluster point of $A$.
	\begin{enumerate}
		\item If $L \in \R$, then $f$ does \textbf{not} have limit $L$ at $c$ if and only if there exists a sequence $(x_n)$ in $A$ with $x_n \neq c$ for all $n \in \N$ such that the sequence $(x_n)$ converges to $c$ but the sequence $(f(x_n))$ does \textbf{not} converge to $L$.

		\item The function $f$ does \textbf{not} have a limit at $c$ if and only if there exists a sequence $(x_n)$ in $A$ with $x_n \neq c$ for all $n \in \N$ such that the sequence $(x_n)$ converges to $c$ but the sequence $(f(x_n))$ does \textbf{not} converge in $\R$.
	\end{enumerate}
\end{theorem}

Let the \textbf{signum function} sgn be defined by
\[\sign (x):=\begin{cases}
		+1 & \text{for }  x>0,   \\
		0  & \text{for }  x=0,   \\
		-1 & \text{for }  x < 0.
	\end{cases}\]
