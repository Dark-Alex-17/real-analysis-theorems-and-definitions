\section{Limit Theorems}

\begin{definition}
	Let $A \subseteq \R$, let $f:A \rightarrow \R$, and let $c \in \R$ be a cluster point of $A$. We say that $f$ is \textbf{bounded on a neighborhood of $c$} if there exists a $\delta$-neighborhood $V_\delta(c)$ of $c$ and a constant $M > 0$ such that we have $|f(x)| \leq M$ for all $x \in A \cap V_\delta (c)$.
\end{definition}

\begin{theorem}
	If $A \subseteq \R$ and $f:A \rightarrow \R$ has a limit at $c \in \R$, then $f$ is bounded on some neighborhood of $c$.
\end{theorem}

\begin{theorem}
	Let $A \subseteq \R$ and let $f$ and $g$ be functions defined on $A$ to $\R$. We  define the \textbf{sum} $f+g$, the \textbf{difference} $f-g$, and the \textbf{product} $fg$ on $A$ to $\R$ to be the functions given by
	\[(f+g)(x):=f(x)+g(x),\]
	\[(f-g)(x):=f(x)-g(x),\]
	\[(fg)(x):=f(x)g(x)\]
	for all $x \in A$. Further, if $b \in \R$, we define the \textbf{multiple} $bf$ to be the function given by
	\[(bf)(x) := bf(x)\ \ \ \text{for all}\ \ \ x \in A\]
	Finally, if $h(x)\neq 0$ for $x \in A$, we define the \textbf{quotient} $f/h$ to be the function given by
	\[\left( \frac{f}{h}\right)(x) := \frac{f(x)}{h(x)}\ \ \ \ \text{for all}\ \ \ \ x \in A\]
\end{theorem}

\begin{theorem}
	let $A \subseteq \R$, let $f$ and $g$ be functions on $A$ to $\R$, and let $c \in \R$ be a cluster point of $A$. Further, let $b \in \R$.
	\begin{enumerate}
		\item If $\lim\limits_{x\to c} f = L$ and $\lim\limits_{x\to c} g = M$, then
		      \[\lim\limits_{x\to c} (f+g) = L+M,\]
		      \[\lim\limits_{x\to c} (f-g)=L-M,\]
		      \[\lim\limits_{x\to c} (fg) = LM,\]
		      \[\lim\limits_{x\to c} (bf) = bL.\]

		\item If $h: A \rightarrow \R$, if $h(x) \neq 0$ for all $x \in A$, and if $\lim\limits_{x\to c} h = H \neq 0$, then
		      \[\lim\limits_{x\to c} \left( \frac{f}{h} \right)= \frac{L}{H}\]
	\end{enumerate}
\end{theorem}

\begin{theorem}
	Let $A \subseteq \R$, let $f: A \rightarrow \R$, and let $c \in \R$ be a cluster point of $A$. If
	\[a \leq f(x) \leq b\ \ \ \ \text{for all}\ \ \ \ x \in A,\ x \neq c,\]
	and if $\lim\limits_{x\to c} f$ exists, then $a \leq \lim\limits_{x\to c} f \leq b$.
\end{theorem}

\begin{theorem}[\textbf{Squeeze Theorem}]
	Let $A \subseteq \R$, let $f,g,h:A \rightarrow \R$, and let $c \in \R$ be a cluster point of $A$. If
	\[f(x) \leq g(x) \leq h(x)\ \ \ \ \text{for all}\ \ \ \ x \in A,\ x \neq c,\]
	and if $\lim\limits_{x\to c} f = L = \lim\limits_{x\to c} h$, then $\lim\limits_{x\to c} g =L$.
\end{theorem}

\begin{theorem}
	Let $A \subseteq \R$, let $f:A \rightarrow \R$ and let $c \in \R$ be a cluster point of $A$. If
	\[\lim\limits_{x\to c} f > 0\ \ \ \left[\textit{respectively, } \lim\limits_{x\to c} f < 0\right],\]
	then there exists a neighborhood $V_\delta (c)$ of $c$ such that $f(x) > 0$ [respectively, $f(x) < 0$] for all $x \in A \cap V_\delta (c),\ x \neq c$.
\end{theorem}
