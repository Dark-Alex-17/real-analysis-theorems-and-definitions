\section{Intervals}

\begin{definition}
	If $a,b \in \R$ satisfy $a<b$, then the \textbf{open interval} determined by $a$ and $b$ is the set
	\[(a,b):= \{x \in \R : a <x < b\}\]

	The points $a$ and $b$ are called the \textbf{endpoints} of the interval.
\end{definition}

\begin{definition}
	If both endpoints $a$ and $b$ are adjoined to an open interval, then we obtain the \textbf{closed interval} determined by $a$ and $b$; namely, the set
	\[[a,b]:=\{x \in \R : a \leq x \leq b\}\]
\end{definition}

\begin{definition}
	The two \textbf{half-open} (or \textbf{half-closed}) intervals determined by $a$ and $b$ are $[a,b)$, which includes the endpoint $a$, and $(a,b]$, which includes the endpoint $b$.
\end{definition}

\begin{definition}
	The \textbf{length} of an interval $(a,b)$ is defined by $b-a$.
\end{definition}

\begin{theorem}[\textbf{Characterization Theorem}]
	If $S$ is a subset of $\R$ that contains at least two points and has the property
	\[\text{if}\ \ \ \ \ x,y \in S\ \ \ \ \ \text{and}\ \ \ \ \ x < y,\ \ \ \ \ \text{then}\ \ \ \ \  [x,y] \subseteq S,\]
	then $S$ is an interval.
\end{theorem}

\begin{theorem}[\textbf{Nested Intervals Property}]
	If $I_n=[a_n,b_n],\ n \in \N$, is a nested sequence of closed bounded intervals, then there exists a number $\xi \in \R$ such that $\xi \in I_n$ for all $n \in \N$.
\end{theorem}

\begin{theorem}
	If $I_n :=[a_n,b_n],\ n \in \N$, is a nested sequence of closed, bounded intervals such that the lengths $b_n-a_n$ of $I_n$ satisfy
	\[\inf \{b_n - a_n : n \in \N\}=0,\]
	then the number $\xi$ contained in $I_n$ for all $n \in \N$ is unique.
\end{theorem}

\begin{theorem}
	The set $\R$ of real numbers is not countable.
\end{theorem}

\begin{theorem}
	The unit interval $[0,1] := \{x \in \R : 0 \leq x \leq 1\}$ is not countable.
\end{theorem}
