\section{Taylor's Theorem}

\begin{theorem}[\textbf{Taylor's Theorem}]
	Let $n \in \N$, let $I:=[a,b]$, and let $f:I \rightarrow \R$ be such that $f$ and its derivative $f', f'', \dots, f^{(n)}$ are continuous on $I$ and that $f^{(n+1)}$ exists on $(a,b)$. If $x_0\in I$, then for any $x$ in $I$ there exists a point $c$ between $x$ and $x_0$ such that
	\[f(x) = f(x_0) + f'(x_0)(x-x_0) + \frac{f''(x_0)}{2!}(x-x_0)^2\]
	\[+ \dots + \frac{f^{(n)}(x_0)}{n!}(x-x_0)^n + \frac{f^{(n+1)}(c)}{(n+1)!}(x-x_0)^{n+1}\]
\end{theorem}

\begin{theorem}
	Let $I$ be an interval, let $x_0$ be an interior point of $I$, and let $n \geq 2$. Suppose that the derivatives $f', f'', \dots, f^{(n)}$ exist and are continuous in a neighborhood of $x_0$ and that $f'(x_0) = \dots = f^{(n-1)(x_0)}$, but $f^{(n)}(x_0) \neq 0$.
	\begin{enumerate}
		\item If $n$ is even and $f^{(n)}(x_0) >0$, then $f$ has a relative minimum at $x_0$.

		\item If $n$ is even and $f^{(n)}<0$, then $f$ has a relative maximum at $x_0$.

		\item If $n$ is odd, then $f$ has neither a relative minimum nor relative maximum at $x_0$.
	\end{enumerate}
\end{theorem}

\begin{definition}
	Let $I \subseteq \R$ be an interval. A function $f:I \rightarrow \R$ is said to be \textbf{convex} on $I$ if for any $t$ satisfying $0 \leq t \leq 1$ and any points $x_1, x_2$ in $I$, we have
	\[f((1-t)x_1+tx_2) \leq (1-t)f(x_1)+tf(x_2).\]
\end{definition}

\begin{theorem}
	Let $I$ be an open interval and let $f: I \rightarrow \R$ have a second derivative on $I$. Then $f$ is a convex function on $I$ if and only if $f''(x) \geq 0$ for all $x \in I$.
\end{theorem}

\begin{theorem}[\textbf{Newton's Method}]
	Let $I:=[a,b]$ and let $f:I \rightarrow \R$ be twice differentiable on $I$. Suppose that $f(a)f(b) < 0$ and that there are constants $m,M$ such that $|f'(x)| \geq m > 0$ and $|f''(x)| \leq M$ for $x \in I$ and let $K:=M/2m$. Then there exists a subinterval $I^*$ containing a zero $r$ of $f$ such that for any $x_1 \in I^*$ the sequence $(x_n)$ defined by
	\[|x_{n+1}-r|\leq x_n - \frac{f(x_n)}{f'(x_n)}\ \ \ \ \text{for all}\ \ \ \ n \in \N,\]

	belongs to $I^*$ and $(x_n)$ converges to $r$. Moreover
	\[|x_{n+1}-r| \leq K|x_n-r|^2\ \ \ \ \text{for all}\ \ \ \ n \in \N.\]
\end{theorem}
