\section{Compact Sets}

\begin{definition}
	Let $A$ be a subset of $\R$. An \textbf{open cover} of $A$ is a collection $\mathcal{G}=\{G_\alpha\}$ of open sets in $\R$ whose union contains $A$; that is,
	\[A \subseteq \bigcup_\alpha G_\alpha\]
	If $\mathcal{G}'$ is a subcollection of sets from $\mathcal{G}$ such that the union of the sets in $\mathcal{G}'$ also contains $A$, then $\mathcal{G}'$ is called a \textbf{subcover} of $\mathcal{G}$. If $\mathcal{G}'$ consists of finitely many sets, then we call $\mathcal{G}'$ a \textbf{finite subcover} of $\mathcal{G}$.
\end{definition}

\begin{definition}
	A subset $K$ of $\R$ is said to be \textbf{compact} if \textit{every} open cover of $K$ has a finite subcover.
\end{definition}

\begin{theorem}
	If $K$ is a compact subset of $\R$, then $K$ is closed and bounded.
\end{theorem}

\begin{theorem}[\textbf{Heine-Borel Theorem}]
	A subset $K$ of $\R$ is compact if and only if it is closed and bounded.
\end{theorem}

\begin{theorem}
	A subset $K$ of $\R$ is compact if and only if every sequence in $K$ has a subsequence that converges to a point in $K$.
\end{theorem}
